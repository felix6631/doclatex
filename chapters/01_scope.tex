\section{Scope}

% Write your translation here.
% Example:
% 이 문서는 C 프로그래밍 언어로 작성된 프로그램의 형식과 해석을 규정한다.
\pnumauto
본 문서는 C 프로그래밍 언어로 작성된 프로그램의 형식과 해석을 규정한다. 
본 문서는 다양한 데이터 처리 시스템에 C 프로그램의 이식성을 제공하기 위해 설계되었다. 
본 문서는 구현자와 프로그래머에 의해 사용되도록 의도되었다.
본 문서는 다음을 제공한다.
\begin{itemize}[label={—}] % Caution: It's em-dash. not hypen.
    \item C 프로그램의 표현;
    \item C 언어의 문법과 제약사항;
    \item C 프로그램 해석의 의미론적 규칙;
    \item C 프로그램에서 처리될 입력 데이터의 표현;
    \item C 프로그램에서 출력될 데이터의 표현;
    \item C의 적합한 구현에 따라 부과된 제한과 한계.
\end{itemize}

\pnumauto
본 문서는 다음을 특정하지 않는다.
\begin{itemize}[label={—}]
    \item C 프로그램이 데이터 처리 시스템에 의해 사용되도록 변환되는 메커니즘;
    \item 데이터 처리 시스템에서 C 프로그램이 호출되는 메커니즘;
    \item C 프로그램에 입력되는 데이터가 변환되는 메커니즘;
    \item C 프로그램에 의해 생성된 출력 데이터가 변환되는 메커니즘;
    \item 특정 데이터 처리 시스템의 용량이나 특정 프로세서의 용량을 초과하는 프로그램 및 그 데이터의 크기나 복잡성;
    \item 적합한 구현을 지원할 수 있는 데이터 처리 시스템의 모든 최소 요구사항.
\end{itemize}

\pnumauto   
부록 J는 C 프로그램이 마주할 수 있는 호환성 이슈에 대한 개요를 제공한다.